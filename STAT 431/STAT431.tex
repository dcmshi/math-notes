\documentclass[english,12pt]{article}

%Packages
\usepackage{amsmath}
\usepackage{amssymb}
\usepackage{amsthm}
\usepackage{fullpage}
\usepackage{hyperref}
\usepackage[normalem]{ulem} %provides uline, uuline and uwave. also sout, xout, dashuline, dotuline. normalem for normal emphasis. otherwise em is underlined instead of italics
\usepackage{enumitem} % Better enumerate. takes in agrs such as resume, [label=(\alph{*})], start=3
\usepackage{mathtools} % needed for \mathclap (underbrace)

%TitleSec to Deal with Paragraph's section break spacing
\usepackage{titlesec}
%\titleformat*{\paragraph}{\bfseries} %amsart paragraph bolding
\titlespacing{\paragraph}
{0pt}{0pt}{1ex}
%
%Indentation
\setlength{\parskip}{\medskipamount}
\setlength{\parindent}{0pt}
\newcommand{\noparskip}{\vspace{-\medskipamount}}

%Other packages
\usepackage{arydshln, nicefrac} % dashed lines in tables, nicefrac
\usepackage{comment} % Commenting: \begin{comment} \end{comment}
\usepackage{cancel} % Math Cancelling: \cancel{whatever} \bcancel neg slope \xcangel both, draw x
\usepackage{relsize} % Enables mathlarger for biggercap, etc: \mathlarger{whatever}

%Theorem environment definitions
\makeatletter
  %Theorems
    \theoremstyle{plain}
    \newtheorem*{theorem}{\protect\theoremname} 
    \newtheorem*{corollary}{\protect\corollaryname}
    \newtheorem*{lemma}{\protect\lemmaname}
    \newtheorem*{proposition}{\protect\propositionname}
    \newtheorem*{namedtheorem}{\namedtheoremname}
      \newcommand{\namedtheoremname}{Theorem} %Doesn't Matter, will get renewed
      \newenvironment{namedthm}[1]{
      \renewcommand{\namedtheoremname}{#1}
      \begin{namedtheorem}}
      {\end{namedtheorem}}
  %Definitions
    \theoremstyle{definition}
    \newtheorem*{definition}{\protect\definitionname}
    \newtheorem*{example}{\protect\examplename}
  %Remarks
    \theoremstyle{definition} %should use remark according to ams.org
    \newtheorem*{remark}{\protect\remarkname}
    \newtheorem*{noteenv}{\protect\notename}
    \newtheorem*{observation}{\protect\observationname}
    \newtheorem*{notationenv}{\protect\notationname}
    \newtheorem*{recallenv}{\protect\recallname}
\makeatother

%Theorem Command Definitions
  \newcommand{\thm}[1]{\begin{theorem} #1 \end{theorem} }
  \newcommand{\cor}[1]{\begin{corollary} #1 \end{corollary} }
  \newcommand{\lem}[1]{\begin{lemma} #1 \end{lemma} }
  \newcommand{\prop}[1]{\begin{proposition} #1 \end{proposition} }
  \newcommand{\nmdthm}[2]{\begin{namedthm}{#1} #2 \end{thm} }
  \newcommand{\defn}[1]{\begin{definition} #1 \end{definition} }
  \newcommand{\eg}[1]{\begin{example} #1 \end{example} }
  \newcommand{\rem}[1]{\begin{remark} #1 \end{remark} }
  \newcommand{\note}[1]{\begin{noteenv} #1 \end{noteenv} }
  \newcommand{\obsrv}[1]{\begin{observation} #1 \end{observation} }
  \newcommand{\notation}[1]{\begin{notationenv} #1 \end{notationenv} }
  \newcommand{\recall}[1]{\begin{recallenv} #1 \end{recallenv} }
  \newcommand{\prf}[1]{\begin{proof} #1 \end{proof} }

\usepackage{babel}
  \providecommand{\theoremname}{Theorem}
  \providecommand{\definitionname}{Definition}
  \providecommand{\corollaryname}{Corollary}
  \providecommand{\lemmaname}{Lemma}
  \providecommand{\propositionname}{Proposition}
  \providecommand{\remarkname}{Remark}
  \providecommand{\notename}{Note}
  \providecommand{\observationname}{Observation}
  \providecommand{\notationname}{Notation}
  \providecommand{\recallname}{Recall}
  \providecommand{\examplename}{Example}

%ams.org
%plain Theorem, Lemma, Corollary, Proposition, Conjecture, Criterion, Algorithm
%definition Definition, Condition, Problem, Example
%remark Remark, Note, Notation, Claim, Summary, Acknowledgment, Case, Conclusion

% Enumerate
  \newcommand{\enum}[1]{\begin{enumerate} #1 \end{enumerate}}
  \newcommand{\enumresume}[1]{\begin{enumerate}[resume*] #1 \end{enumerate}} %resume enumerate after interuption
  \newcommand{\enuma}[1]{\begin{enumerate}[label=(\alph{*})] #1 \end{enumerate}}
  \newcommand{\enumA}[1]{\begin{enumerate}[label=(\Alph{*})] #1 \end{enumerate}}
  \newcommand{\enumi}[1]{\begin{enumerate}[label=(\roman{*})] #1 \end{enumerate}}
  \newcommand{\enumI}[1]{\begin{enumerate}[label=(\Roman{*})] #1 \end{enumerate}}

% Integrals
\newcommand{\dt}{\mbox{ dt}}
\newcommand{\dx}{\mbox{ dx}}
\newcommand{\ds}{\mbox{ ds}}

% Norms, underbrace, floor, ceiling, inner product
\newcommand{\norm}[1]{\left\lVert #1 \right\rVert}
\newcommand{\abs}[1]{\left\lvert #1 \right\rvert}
\newcommand{\floor}[1]{\left\lfloor #1 \right\rfloor}
\newcommand{\ceil}[1]{\left\lceil #1 \right\rceil}
\newcommand{\ubrace}[1]{\underbrace{#1}}
\newcommand{\obrace}[1]{\overbrace{#1}}
\newcommand{\oline}[1]{\overline{#1}}
\newcommand{\inprod}[1]{\left\langle #1 \right\rangle} %Modified from Eeshan's preamble
\newcommand{\clapubrace}[2]{\underbrace{#1}_{\mathclap{#2}}}
\newcommand{\clapobrace}[2]{\overbrace{#1}^{\mathclap{#2}}}

% Big operators and i sums, unions, intersections
\newcommand{\inter}{\bigcap}
\newcommand{\union}{\bigcup}
\newcommand{\iinter}[3]{\bigcap_{i=#1}^{#2} #3} % Modified from Eeshan's preamble
\newcommand{\isum}[3]{\bigsum_{i=#1}^{#2} #3} % Modified from Eeshan's preamble
\newcommand{\iuni}[3]{\bigcup_{i=#1}^{#2} #3} % Modified from Eeshan's preamble

%Brackets
\newcommand{\custbrac}[3]{\left#1#3\right#2} % Custom Brackets, . for no brac
\newcommand{\brac}[1]{\left(#1\right)} % Parenthesis
\newcommand{\sqbrac}[1]{\left[#1\right]} % Square Brackets
\newcommand{\curlybrac}[1]{\left\{#1\right\}} % Curly Brackets

%Spaces
\newcommand{\R}{\mathbb{R}} % Reals
\newcommand{\Z}{\mathbb{Z}} % Integers
\newcommand{\N}{\mathbb{N}} % Naturals
\newcommand{\Q}{\mathbb{Q}} % Rationals
\newcommand{\C}{\mathbb{C}} % Complex
\newcommand{\T}{\mathbb{T}} % circle, 1 dimensional Torus
\newcommand{\I}{\mathbb{I}} % Unit interval
\newcommand{\bb}[1]{\mathbb{#1}} % For other Blackboard letters

%Script
\newcommand{\sP}{\mathcal{P}} % Partion P
\newcommand{\sQ}{\mathcal{Q}} % Partion Q
\newcommand{\sB}{\mathcal{B}} % Borel
\newcommand{\sF}{\mathcal{F}} % Family of functions
\newcommand{\sH}{\mathcal{H}} % Hilbert Space
\newcommand{\scr}[1]{\mathcal{#1}} % For other Script fonts

%Analysis Stuff
\newcommand{\lms}{\lambda^{*}} % lambda star, outer measure
\newcommand{\linfty}[1] {\lim _{#1 \to \infty}} %Modified from Eeshan's preamble


%Tikz Decocations, graphics
\usepackage{tikz, graphicx} % Note graphicx needed for includegraphics
\usetikzlibrary{decorations.pathreplacing}

\newcommand{\startbrace}[1]{\tikz[remember picture] \node[coordinate,yshift=0.5em] (#101) {};}
\newcommand{\finishbrace}[2]{\tikz[remember picture] \node[coordinate] (#102) {};
\begin{tikzpicture}[overlay,remember picture]
      \path (#102) -| node[coordinate] (#103) {} (#101); %Creates a node3 vertically down from 1

      \draw[thick,decorate,decoration={brace,amplitude=3pt}]
            (#101) -- (#103) node[midway, right=4pt] {#2}; %2 is Text
  \end{tikzpicture}} % Use: \startbrace{label}, \finishbrace{label}{text}

% Othercommands
\newcommand{\lect}[2]{\flushleft \emph{#1 \hfill #2}}
\newcommand{\bmat}[1]{\begin{bmatrix}#1\end{bmatrix}}
\newcommand{\x}{\times}
\newcommand{\cd}{\cdot}

% Probability commands
\newcommand{\p}[1]{\mbox{P} \left( #1 \right)}
\newcommand{\ex}[1]{\mbox{E} \left[ #1 \right]}
\newcommand{\var}[1]{\mbox{Var} \left( #1 \right)}
\newcommand{\cov}[1]{\mbox{Cov} \left( #1 \right)}
\newcommand{\condp}[2]{\mbox{E} \left( \left. #1 \ \right\vert \left. #2 \right. \right)}
\newcommand{\condex}[2]{\mbox{E} \left[ \left. #1 \ \right\vert \left. #2 \right. \right]}
\newcommand{\condvar}[2]{\mbox{Var} \left( \left. #1 \ \right\lvert \left. #2 \right. \right)}
\begin{document}
\begin{titlepage}

\newcommand{\HRule}{\rule{\linewidth}{0.5mm}} % Defines a new command for the horizontal lines, change thickness here

\center % Center everything on the page

\textsc{\LARGE University of Waterloo}\\[1.5cm] % Name of your university/college
\textsc{\Large STAT 431}\\[0.5cm] % Major heading such as course name
\textsc{\large Generalized Linear Models}\\[0.5cm] % Minor heading such as course title

\HRule \\[0.4cm]
{ \huge \bfseries Course Notes}\\[0.4cm] % Title of your document
\HRule \\[1.5cm]

\begin{minipage}{0.4\textwidth}
\begin{flushleft} \large
\emph{Author:}\\
David \textsc{Shi} % Your name
\end{flushleft}
\end{minipage}
~
\begin{minipage}{0.4\textwidth}
\begin{flushright} \large
\emph{Professor:} \\
Dr. Cecilia \textsc{Cotton} % Supervisor's Name
\end{flushright}
\end{minipage}\\[4cm]

{\large \today}\\[3cm] % Date, change the \today to a set date if you want to be precise

%\includegraphics{Logo}\\[1cm] % Include a department/university logo - this will require the graphicx package

\vfill % Fill the rest of the page with whitespace

\end{titlepage}

\tableofcontents % Include a table of contents

\newpage


\section*{Lecture 1}
\section{Review}
Recall from STAT 331: Applied Linear Models

THERE'S A REGRESSION PICTURE HERE

We want to ask the question, what is the nature of association between age and blood pressure for example.  where $y=$blood pressure and $x=$age.  In this case the response $y$ is a continuous variable and we are trying to fit the straight regression model:
\[y=\beta_0+\beta_1x+\varepsilon_i\qquad \varepsilon_i\sim N(0,\sigma^2)\qquad y\sim N(\beta_0+\beta_1x,\sigma^2)\]

\begin{enumerate}
\item We estimate using least squares by minimizing the quantity:
\[\sum e_i^2=\sum(y_i-\beta_0-\beta_1x_i)^2\]
\item Then we try to make some inference about the $\beta$ parameters
\begin{itemize}
\item confidence intervals for $\beta_0$ and $\beta_1$
\item hypothesis testing $H_0:\beta_1=0$
\end{itemize}
\item Also we can do model checking, to see if our model is appropriate given the assumptions based on the residuals
\item We can also construct predictions for $\hat y_i$ given a new $x_i$.
\item This can be extended to multiple linear regression:
\[y_i=\beta_0+\beta_1x_{i1}+\beta_2x_{i2}+\ldots+\varepsilon_i\]
\end{enumerate}

Other topics of interest my include:
\begin{itemize}
\item Variance Stabilizing Transformations
\item Matrix Notation $Y=X\beta$
\item ANOVA Models (Analysis of Variance)
\end{itemize}

Here we will consider non-normal response data (i.e. non-continuous)

\begin{enumerate}
\item Binary Response (outcome $y=0\text{ or }1$) (i.e. dead/alive following treatment)
\item Poisson Response ($y\in \mathbb{Z}^+$) (i..e. number of claims on an insurance policy)
\item Other distributions that belong to the exponential family.
\end{enumerate}

\section{Likelihood Methods}
Material covered is from Appendix A of the course notes and Appendix A of the textbook.

\eg{
There was a study of toxicity of a certain poison on beetles.  10 beetles were exposed to the poison and 6 of those exposed beetles survived.


Let $y_i$ be the survival status of each beetle $i=1\ldots 10$.
\[y_i=\begin{cases}
1 & \text{beetle }i \text{ is alive}\\
0 & \text{beetle }i \text{ is dead}\\
\end{cases}\]

Assume:
\[P(y_i=0)=\theta\qquad y_i\sim \text{Bernoulli}(\theta)\qquad y_1,y_2,\ldots,y_n\text{ are independent and identically distributed}\] 
then we have that:
\[y=\sum_{i=1}^ny_i=\text{total number of surviving beetles}\qquad y\sim\text{Binomial}(n,\theta)\qquad \theta=\text{probability of surviving}\]

We want to conduct inference on the parameter $\theta$.  Our three goals are:
\begin{enumerate}
\item Estimate $\theta$
\item Carry out hypothesis test $H_0=0.50$
\item Quantify how good our estimate is (confidence intervals)
\end{enumerate}

Define the following quantities:
\begin{enumerate}
\item Likelihood Function
\[L(\theta;y)=c\cdot P(Y=y\mid\theta))\]
\begin{itemize}
\item Function is proportional by a constant factor $c$ to the likelihood probability of observing the data you actually obtained.
\item Ranks all possible $\theta$  in terms of consistency with observed $y$.
\item Want to find $\theta$ that makes the observable data the most likely (Maximum Likelihood).  So we find the $\theta$ that maximizes $L(\theta;y)$.  However it is usually easier to maximize $\ln(L(\theta;y))=l(\theta;y)$, since $\ln(\cdot)$ is a 1-1 function so maximum should be preserved.
\end{itemize}
\item Log Likelihood Function 
\[\ln(L(\theta;y))=l(\theta;y)\]

Take for example the log likelihood of the binomial distribution:
\begin{align*}
L(\theta;y)
&=cP(Y=y;\theta)\\
&=c{n\choose y} \theta^y(1-\theta)^{n-y} \qquad \text{let }c=\frac{1}{{n\choose y}}\\
&=\theta^y(1-\theta)^{n-y}\\
l(\theta;y)&=\ln(L(\theta;y))
=\ln(\theta^y(1-\theta)^{n-y})
=y\ln(\theta)+\frac{n-y}{\ln(1-\theta)}
\end{align*}

Maximizing $l(\theta;y)$ is our goal, so we differentiate with respect to $\theta$ and set the derivative equal to 0.

\item Score Function
\[S(\theta;y)=l'(\theta;y)=\frac{\partial}{\partial\theta}l(\theta;y)\]

From the binomial example we can derive the score function and the appropiate estimate for $\theta$:
\begin{align*}
S(\theta;y)&=\frac{y}{\theta} -\frac{n-y}{1-\theta} \qquad \text{set }S(\hat\theta; y)=0\\
0&=\frac{y}{\hat\theta} -\frac{n-y}{1-\hat\theta}\\
0&=y(1-\hat\theta)-\hat\theta(n-y)\\
0&= y-y\hat\theta-\hat\theta n+\hat\theta y\\
0&=y-\hat\theta n\\
\rightarrow \hat\theta=\frac{y}{n}
\end{align*}

Therefore the maximum likelihood estimator (MLE) is the observed proportion of those beetles that survived.

Recall from calculs that we should check that this critical point is indeed a maximum with the second derivative test.

\item Information Function:
\[I(\theta;y)=l''(\theta;y)=-S'(\theta;y)\]

The information function allows us to easily test the property of critical points from the log-likelihood function, if $I(\theta;y)>0\rightarrow \hat\theta$ is a maximum.  Consider the information function for the binomial:
\begin{align*}
I(\theta;y)&=-\frac{\partial}{\partial \theta}\left(\frac{y}{\theta}-\frac{n-y}{1-\theta}\right)
=-\left(-\frac{y}{\theta^2}-\frac{n-y}{(1-\theta)^2}\right)
=\frac{y}{\theta^2}+\frac{n-y}{(1-\theta)^2}
\end{align*}

Since $n\ge y$ we have that $I(\theta;y)>0$, therefore $\hat\theta$ is a maximum.
\end{enumerate}
}

We can now estimate $\theta$ with $\hat\theta$:
\[n=10\qquad y=6\qquad\rightarrow\qquad  \hat\theta=\frac{6}{10}=0.6\]

\section{Newton Raphson}
In this case, solving $S(\hat\theta;y)=0$ was quite easy, but this won't always be the case.  Here we consider an interative method, based on Taylor Series:
\[f(x)=f(a)+\frac{f'(a)(x-a)}{1!}+\frac{f''(a)(x-a)^2}{2!}+\ldots\]

We want to find $\hat\theta$ which is a root of $S(\theta;y)$, so we make an educated guess of $\hat\theta$, say some $\theta_0$ that we think is close, then we use the Taylor expansion of $S(\theta;y)$ about $\theta_0$.
\[S(\theta)=S(\theta_0)+S'(\theta_0)(\theta-\theta_0)+\frac{S''(\theta_0)(\theta-\theta_0)^2}{2!}+\ldots\]

We assume that $|\theta-\theta_0|$ is small, and we only use the first order term in the expansion:
\begin{align*}
S(\theta)\approx S(\theta_0) +S'(\theta_0)(\theta-\theta_0)
\end{align*}

\section*{Lecture 2}
\section*{Lecture 3}
\section*{Lecture 4}
\section*{Lecture 5}
\section*{Lecture 6}
\section*{Lecture 7}
\section*{Lecture 8}
\section*{Lecture 9}
\section*{Lecture 10}
\section*{Lecture 11}
\section*{Lecture 12}
\section*{Lecture 13}
\section*{Lecture 14}
\section*{Lecture 15}
\section*{Lecture 16}
\section*{Lecture 17}
\section*{Lecture 18}
\section*{Lecture 19}
\section*{Lecture 20}

\end{document}
