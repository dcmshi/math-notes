\documentclass[english,12pt]{article}

%Packages
\usepackage{amsmath}
\usepackage{amssymb}
\usepackage{amsthm}
\usepackage{fullpage}
\usepackage[normalem]{ulem} %provides uline, uuline and uwave. also sout, xout, dashuline, dotuline. normalem for normal emphasis. otherwise em is underlined instead of italics
\usepackage{enumitem} % Better enumerate. takes in agrs such as resume, [label=(\alph{*})], start=3
\usepackage{mathtools} % needed for \mathclap (underbrace)

%TitleSec to Deal with Paragraph's section break spacing
\usepackage{titlesec}
%\titleformat*{\paragraph}{\bfseries} %amsart paragraph bolding
\titlespacing{\paragraph}
{0pt}{0pt}{1ex}

%Indentation
\setlength{\parskip}{\medskipamount}
\setlength{\parindent}{0pt}
\newcommand{\noparskip}{\vspace{-\medskipamount}}

%Other packages
\usepackage{arydshln, nicefrac} % dashed lines in tables, nicefrac
\usepackage{comment} % Commenting: \begin{comment} \end{comment}
\usepackage{cancel} % Math Cancelling: \cancel{whatever} \bcancel neg slope \xcangel both, draw x
\usepackage{relsize} % Enables mathlarger for biggercap, etc: \mathlarger{whatever}
\usepackage{hyperref}

%Theorem environment definitions
\makeatletter
  %Theorems
    \theoremstyle{plain}
    \newtheorem*{theorem}{\protect\theoremname} 
    \newtheorem*{corollary}{\protect\corollaryname}
    \newtheorem*{lemma}{\protect\lemmaname}
    \newtheorem*{proposition}{\protect\propositionname}
    \newtheorem*{namedtheorem}{\namedtheoremname}
      \newcommand{\namedtheoremname}{Theorem} %Doesn't Matter, will get renewed
      \newenvironment{namedthm}[1]{
      \renewcommand{\namedtheoremname}{#1}
      \begin{namedtheorem}}
      {\end{namedtheorem}}
  %Definitions
    \theoremstyle{definition}
    \newtheorem*{definition}{\protect\definitionname}
    \newtheorem*{example}{\protect\examplename}
  %Remarks
    \theoremstyle{definition} %should use remark according to ams.org
    \newtheorem*{remark}{\protect\remarkname}
    \newtheorem*{noteenv}{\protect\notename}
    \newtheorem*{observation}{\protect\observationname}
    \newtheorem*{notationenv}{\protect\notationname}
    \newtheorem*{recallenv}{\protect\recallname}
\makeatother

%Theorem Command Definitions
  \newcommand{\thm}[1]{\begin{theorem} #1 \end{theorem} }
  \newcommand{\cor}[1]{\begin{corollary} #1 \end{corollary} }
  \newcommand{\lem}[1]{\begin{lemma} #1 \end{lemma} }
  \newcommand{\prop}[1]{\begin{proposition} #1 \end{proposition} }
  \newcommand{\nmdthm}[2]{\begin{namedthm}{#1} #2 \end{thm} }
  \newcommand{\defn}[1]{\begin{definition} #1 \end{definition} }
  \newcommand{\eg}[1]{\begin{example} #1 \end{example} }
  \newcommand{\rem}[1]{\begin{remark} #1 \end{remark} }
  \newcommand{\note}[1]{\begin{noteenv} #1 \end{noteenv} }
  \newcommand{\obsrv}[1]{\begin{observation} #1 \end{observation} }
  \newcommand{\notation}[1]{\begin{notationenv} #1 \end{notationenv} }
  \newcommand{\recall}[1]{\begin{recallenv} #1 \end{recallenv} }
  \newcommand{\prf}[1]{\begin{proof} #1 \end{proof} }

\usepackage{babel}
  \providecommand{\theoremname}{Theorem}
  \providecommand{\definitionname}{Definition}
  \providecommand{\corollaryname}{Corollary}
  \providecommand{\lemmaname}{Lemma}
  \providecommand{\propositionname}{Proposition}
  \providecommand{\remarkname}{Remark}
  \providecommand{\notename}{Note}
  \providecommand{\observationname}{Observation}
  \providecommand{\notationname}{Notation}
  \providecommand{\recallname}{Recall}
  \providecommand{\examplename}{Example}

%ams.org
%plain Theorem, Lemma, Corollary, Proposition, Conjecture, Criterion, Algorithm
%definition Definition, Condition, Problem, Example
%remark Remark, Note, Notation, Claim, Summary, Acknowledgment, Case, Conclusion

% Enumerate
  \newcommand{\enum}[1]{\begin{enumerate} #1 \end{enumerate}}
  \newcommand{\enumresume}[1]{\begin{enumerate}[resume*] #1 \end{enumerate}} %resume enumerate after interuption
  \newcommand{\enuma}[1]{\begin{enumerate}[label=(\alph{*})] #1 \end{enumerate}}
  \newcommand{\enumA}[1]{\begin{enumerate}[label=(\Alph{*})] #1 \end{enumerate}}
  \newcommand{\enumi}[1]{\begin{enumerate}[label=(\roman{*})] #1 \end{enumerate}}
  \newcommand{\enumI}[1]{\begin{enumerate}[label=(\Roman{*})] #1 \end{enumerate}}

% Integrals
\newcommand{\dt}{\mbox{ dt}}
\newcommand{\dx}{\mbox{ dx}}
\newcommand{\ds}{\mbox{ ds}}

% Norms, underbrace, floor, ceiling, inner product
\newcommand{\norm}[1]{\left\lVert #1 \right\rVert}
\newcommand{\abs}[1]{\left\lvert #1 \right\rvert}
\newcommand{\floor}[1]{\left\lfloor #1 \right\rfloor}
\newcommand{\ceil}[1]{\left\lceil #1 \right\rceil}
\newcommand{\ubrace}[1]{\underbrace{#1}}
\newcommand{\obrace}[1]{\overbrace{#1}}
\newcommand{\oline}[1]{\overline{#1}}
\newcommand{\inprod}[1]{\left\langle #1 \right\rangle} %Modified from Eeshan's preamble
\newcommand{\clapubrace}[2]{\underbrace{#1}_{\mathclap{#2}}}
\newcommand{\clapobrace}[2]{\overbrace{#1}^{\mathclap{#2}}}

% Big operators and i sums, unions, intersections
\newcommand{\inter}{\bigcap}
\newcommand{\union}{\bigcup}
\newcommand{\iinter}[3]{\bigcap_{i=#1}^{#2} #3} % Modified from Eeshan's preamble
\newcommand{\isum}[3]{\bigsum_{i=#1}^{#2} #3} % Modified from Eeshan's preamble
\newcommand{\iuni}[3]{\bigcup_{i=#1}^{#2} #3} % Modified from Eeshan's preamble

%Brackets
\newcommand{\custbrac}[3]{\left#1#3\right#2} % Custom Brackets, . for no brac
\newcommand{\brac}[1]{\left(#1\right)} % Parenthesis
\newcommand{\sqbrac}[1]{\left[#1\right]} % Square Brackets
\newcommand{\curlybrac}[1]{\left\{#1\right\}} % Curly Brackets

%Spaces
\newcommand{\R}{\mathbb{R}} % Reals
\newcommand{\Z}{\mathbb{Z}} % Integers
\newcommand{\N}{\mathbb{N}} % Naturals
\newcommand{\Q}{\mathbb{Q}} % Rationals
\newcommand{\C}{\mathbb{C}} % Complex
\newcommand{\T}{\mathbb{T}} % circle, 1 dimensional Torus
\newcommand{\I}{\mathbb{I}} % Unit interval
\newcommand{\bb}[1]{\mathbb{#1}} % For other Blackboard letters

%Script
\newcommand{\sP}{\mathcal{P}} % Partion P
\newcommand{\sQ}{\mathcal{Q}} % Partion Q
\newcommand{\sB}{\mathcal{B}} % Borel
\newcommand{\sF}{\mathcal{F}} % Family of functions
\newcommand{\sH}{\mathcal{H}} % Hilbert Space
\newcommand{\scr}[1]{\mathcal{#1}} % For other Script fonts

%Analysis Stuff
\newcommand{\lms}{\lambda^{*}} % lambda star, outer measure
\newcommand{\linfty}[1] {\lim _{#1 \to \infty}} %Modified from Eeshan's preamble


%Tikz Decocations, graphics
\usepackage{tikz, graphicx} % Note graphicx needed for includegraphics
\usetikzlibrary{decorations.pathreplacing}

\newcommand{\startbrace}[1]{\tikz[remember picture] \node[coordinate,yshift=0.5em] (#101) {};}
\newcommand{\finishbrace}[2]{\tikz[remember picture] \node[coordinate] (#102) {};
\begin{tikzpicture}[overlay,remember picture]
      \path (#102) -| node[coordinate] (#103) {} (#101); %Creates a node3 vertically down from 1

      \draw[thick,decorate,decoration={brace,amplitude=3pt}]
            (#101) -- (#103) node[midway, right=4pt] {#2}; %2 is Text
  \end{tikzpicture}} % Use: \startbrace{label}, \finishbrace{label}{text}

% Othercommands
\newcommand{\lect}[2]{\flushleft \emph{#1 \hfill #2}}
\newcommand{\bmat}[1]{\begin{bmatrix}#1\end{bmatrix}}
\newcommand{\x}{\times}
\newcommand{\cd}{\cdot}

% Probability commands
\newcommand{\p}[1]{\mbox{P} \left( #1 \right)}
\newcommand{\ex}[1]{\mbox{E} \left[ #1 \right]}
\newcommand{\var}[1]{\mbox{Var} \left( #1 \right)}
\newcommand{\cov}[1]{\mbox{Cov} \left( #1 \right)}
\newcommand{\condp}[2]{\mbox{E} \left( \left. #1 \ \right\vert \left. #2 \right. \right)}
\newcommand{\condex}[2]{\mbox{E} \left[ \left. #1 \ \right\vert \left. #2 \right. \right]}
\newcommand{\condvar}[2]{\mbox{Var} \left( \left. #1 \ \right\lvert \left. #2 \right. \right)}


\begin{document}
\begin{titlepage}

\newcommand{\HRule}{\rule{\linewidth}{0.5mm}} % Defines a new command for the horizontal lines, change thickness here

\center % Center everything on the page

\textsc{\LARGE University of Waterloo}\\[1.5cm] % Name of your university/college
\textsc{\Large MATH 239}\\[0.5cm] % Major heading such as course name
\textsc{\large Introduction to Combinatorics}\\[0.5cm] % Minor heading such as course title

\HRule \\[0.4cm]
{ \huge \bfseries Course Notes}\\[0.4cm] % Title of your document
\HRule \\[1.5cm]

\begin{minipage}{0.4\textwidth}
\begin{flushleft} \large
\emph{Author:}\\
David \textsc{Shi}\\ % Your name
20339941
\end{flushleft}
\end{minipage}
~
\begin{minipage}{0.4\textwidth}
\begin{flushright} \large
\emph{Professor:} \\
Dr. Timothy \textsc{Chan} % Supervisor's Name
\end{flushright}
\end{minipage}\\[4cm]

{\large \today}\\[3cm] % Date, change the \today to a set date if you want to be precise

%\includegraphics{Logo}\\[1cm] % Include a department/university logo - this will require the graphicx package

\vfill % Fill the rest of the page with whitespace

\end{titlepage}


\newpage
\section{Lecture 1}
\begin{itemize}
\item Binomial coefficients
\item Bijections
\end{itemize}

\[{n\choose k} = \frac{n!}{(n-k)!k!}\qquad \text{Binomial coefficient}\]
comes from the Binomial Theorem
\[(1+x)^n=\sum_{k=0}^n{n\choose k}x^k\]

\subsection{Combinatorial Explanation of the coefficient}
\[(1+x)^n=\underbrace{(1+x)(1+x)\ldots(1+x)}_{n\text{ times}}\]
\begin{itemize}
\item In the expansion, we take one term from each binomial and multiply to get one possible term in the expansion. (sum over all such terms)
\[(1+x)(1+x) = (1)(1) + (1)(x)+(1)(x)+(x)(x) = 1+2x+x^2\]
\item we get $x^k$ by selecting $k$ copies of $x$ and $n-k$ copies of $1$.  The coefficient of $x^k$ is the number of ways of selecting $k$ copies of $x$ which is ${n\choose k}$
\end{itemize}
This gives the binomial theorem.

\subsection{Bijections}
A function $f:S\to T$ is:
\begin{enumerate}
\item 1-1 (injective) if no two of the same elements of $S$ are mapped to teh same element in $T$.
\item onto (surjective) if every element in $T$ has something from $S$ that maps to it
\item bijection both injective and surjective 
\end{enumerate}

Suppose $S$ and $T$ are finite, and $F:S\to T$ is a function
\begin{enumerate}
\item If $f$ is 1-1, then $|T|\ge |S|$
\item If $f$ is onto, then $|T|\le |S|$
\item If $f$ is bijective, then $|T| = |S|$
\end{enumerate}

\begin{example}
$A=\{a,b,c\}$ and $B=\{1,2,3\}$.  We define $f:A\to B$ where $f(a)=2$, $f(b)=3$, $f(c)=1$.  Then $f$ is both onto and 1-1, therefore it is a bijection
\end{example}

\begin{example}
$S=$ all subsets of $\{1,\ldots, n\}$ of size $k$, and $T=$ all subsets of $\{1,\ldots, n\}$ of size $n-k$.  Find a bijection of $f:S\to T$.

Consider a smaller example where $n=4$ and $k=1$, so $S=\{\{1\},\{2\},\{3\},\{4\}\}$ and $T=\{\{1,2,3\},\{1,2,4\},\{1,3,4\},\{2,3,4\}\}$

Following from the example we define a function $f:S\to T$ by $f(x)=\{1\ldots, n\}\setminus x$ for each $x\in S$.  We check if $f(x)\in T$.  Since $|x|=k$ and $x\subseteq \{1,\ldots, n\}$, we see that $|f(x)|=|\{1,\ldots, n\}\setminus x| = n-k$.  So $f(x)\in T$.  Though we've shown that $f(x)$ maps elements from $S$ to $T$, it's quite hard to prove that it's onto/1-1 so we will introduce the concept of an inverse, and find that instead. 
\end{example}

\begin{definition}
Inverse: a function $f^{-1}:T\to S$ (of $f:S\to T$) such that for all $x\in S$, $f^{-1}(f(x)) = x$ and for all $y\in T$, $f(f^{-1}(y))$.  So intuitively the inverse just ``reverses'' the action of $f$.
\end{definition}

\begin{theorem}
$f$ is a bijection iff $f$ has an inverse
\end{theorem}

\begin{proof}
If $f$ is 1-1, and not onto, then $f$ cannot have an inverse.  Similarly if $f$ is onto, and not 1-1, then $f$ cannot have an inverse
\end{proof}

\begin{example}
Back to the example.  The inverse of $f$ is $f^{-1}:T\to S$ where $f^{-1}(y)=\{1,\ldots, n\}\setminus y$ for all $y\in T$.

Justification of $f^{-1}$, for each $x\in S$,
\begin{align*}
f^{-1}(f(x)) &= f^{-1}(\{1,\ldots, n\}\setminus x)
=\{1,\ldots, n\}\setminus (\{1,\ldots, n\}\setminus x)
=x
\end{align*}
\end{example}

\end{document}